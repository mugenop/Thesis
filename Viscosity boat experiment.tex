%% Based on a TeXnicCenter-Template by Tino Weinkauf.
%%%%%%%%%%%%%%%%%%%%%%%%%%%%%%%%%%%%%%%%%%%%%%%%%%%%%%%%%%%%%

%%%%%%%%%%%%%%%%%%%%%%%%%%%%%%%%%%%%%%%%%%%%%%%%%%%%%%%%%%%%%
%% HEADER
%%%%%%%%%%%%%%%%%%%%%%%%%%%%%%%%%%%%%%%%%%%%%%%%%%%%%%%%%%%%%
\documentclass[a4paper,twoside,10pt]{report}
% Alternative Options:
%	Paper Size: a4paper / a5paper / b5paper / letterpaper / legalpaper / executivepaper
% Duplex: oneside / twoside
% Base Font Size: 10pt / 11pt / 12pt


%% Language %%%%%%%%%%%%%%%%%%%%%%%%%%%%%%%%%%%%%%%%%%%%%%%%%
\usepackage[USenglish]{babel} %francais, polish, spanish, ...
\usepackage[T1]{fontenc}
\usepackage[ansinew]{inputenc}

\usepackage{lmodern} %Type1-font for non-english texts and characters


%% Packages for Graphics & Figures %%%%%%%%%%%%%%%%%%%%%%%%%%
\usepackage{graphicx} %%For loading graphic files
%\usepackage{subfig} %%Subfigures inside a figure
%\usepackage{pst-all} %%PSTricks - not useable with pdfLaTeX

%% Please note:
%% Images can be included using \includegraphics{Dateiname}
%% resp. using the dialog in the Insert menu.
%% 
%% The mode "LaTeX => PDF" allows the following formats:
%%   .jpg  .png  .pdf  .mps
%% 
%% The modes "LaTeX => DVI", "LaTeX => PS" und "LaTeX => PS => PDF"
%% allow the following formats:
%%   .eps  .ps  .bmp  .pict  .pntg


%% Math Packages %%%%%%%%%%%%%%%%%%%%%%%%%%%%%%%%%%%%%%%%%%%%
\usepackage{amsmath}
\usepackage{amsthm}
\usepackage{amsfonts}


%% Line Spacing %%%%%%%%%%%%%%%%%%%%%%%%%%%%%%%%%%%%%%%%%%%%%
%\usepackage{setspace}
%\singlespacing        %% 1-spacing (default)
%\onehalfspacing       %% 1,5-spacing
%\doublespacing        %% 2-spacing


%% Other Packages %%%%%%%%%%%%%%%%%%%%%%%%%%%%%%%%%%%%%%%%%%%
%\usepackage{a4wide} %%Smaller margins = more text per page.
%\usepackage{fancyhdr} %%Fancy headings
%\usepackage{longtable} %%For tables, that exceed one page


%%%%%%%%%%%%%%%%%%%%%%%%%%%%%%%%%%%%%%%%%%%%%%%%%%%%%%%%%%%%%
%% Remarks
%%%%%%%%%%%%%%%%%%%%%%%%%%%%%%%%%%%%%%%%%%%%%%%%%%%%%%%%%%%%%
%
% TODO:
% 1. Edit the used packages and their options (see above).
% 2. If you want, add a BibTeX-File to the project
%    (e.g., 'literature.bib').
% 3. Happy TeXing!
%
%%%%%%%%%%%%%%%%%%%%%%%%%%%%%%%%%%%%%%%%%%%%%%%%%%%%%%%%%%%%%

%%%%%%%%%%%%%%%%%%%%%%%%%%%%%%%%%%%%%%%%%%%%%%%%%%%%%%%%%%%%%
%% Options / Modifications
%%%%%%%%%%%%%%%%%%%%%%%%%%%%%%%%%%%%%%%%%%%%%%%%%%%%%%%%%%%%%

%\input{options} %You need a file 'options.tex' for this
%% ==> TeXnicCenter supplies some possible option files
%% ==> with its templates (File | New from Template...).



%%%%%%%%%%%%%%%%%%%%%%%%%%%%%%%%%%%%%%%%%%%%%%%%%%%%%%%%%%%%%
%% DOCUMENT
%%%%%%%%%%%%%%%%%%%%%%%%%%%%%%%%%%%%%%%%%%%%%%%%%%%%%%%%%%%%%
\begin{document}

\pagestyle{empty} %No headings for the first pages.


%% Title Page %%%%%%%%%%%%%%%%%%%%%%%%%%%%%%%%%%%%%%%%%%%%%%%
%% ==> Write your text here or include other files.

%% The simple version:
\title{Experiment protocol}
\author{Adel Djellouli}
%\date{} %%If commented, the current date is used.
\maketitle

%% The nice version:
%\input{titlepage} %%You need a file 'titlepage.tex' for this.
%% ==> TeXnicCenter supplies a possible titlepage file
%% ==> with its templates (File | New from Template...).


%% Inhaltsverzeichnis %%%%%%%%%%%%%%%%%%%%%%%%%%%%%%%%%%%%%%%
\tableofcontents %Table of contents
\cleardoublepage %The first chapter should start on an odd page.

\pagestyle{plain} %Now display headings: headings / fancy / ...



%% Chapters %%%%%%%%%%%%%%%%%%%%%%%%%%%%%%%%%%%%%%%%%%%%%%%%%
%% ==> Write your text here or include other files.

%\input{intro} %You need a file 'intro.tex' for this.


%%%%%%%%%%%%%%%%%%%%%%%%%%%%%%%%%%%%%%%%%%%%%%%%%%%%%%%%%%%%%
%% ==> Some hints are following:

\chapter{Boat experiment varying the viscosity, the geometric properties and material properties}

\section{Introduction}
\subsection{Purpose}
\paragraph{}
The purpose of this experiment is to investigate the swimming of spherical shells according to their geometric/material properties and in different rheological configurations.
\subsection{Experimental setup}
\paragraph{}
The experimental setup to be used is a frictionless rail to which we attach a rigid support that allows to keep the shell immersed under liquid surface and direct the buckling spot as to get a parallel motion to the rail.
\paragraph{Note:}
This direction has to be validated as being the most efficient, this is why a perpendicular direction will be investigated also.

\newpage
\section{Modus operandi}
\subsection{Preliminary setup}
Different verifications/installations have to be made:

\begin{enumerate}
	\item Fix the position of the rail by using on one end a fixed support to which we add a guide for stable positioning.
	\item Add an elevator to tune the horizontality of the rail, measured by a ``niveau a bulle'', also supplied with positioning guides.
	\item Ensure that the setup has a fixed position by linking all the components together.
	\item Verify the stability and the non flexure of the support during the buckling.
\end{enumerate}
\newpage
\subsection{Experiment protocol}

\paragraph{}
Three thicknesses will be used in 7 different viscosity conditions.\\
Two material shells will also be studied.\\
For each liquid instance, this is the protocol to be followed:

\begin{enumerate}
	\item Open off the rail pressure inlet.
	\item Check the orthogonality of the ball by making it buckle once.
  \item Check the horizontality of the rail.
	
	
	\item Measure the temperature of the liquid.
	\item Perform a 20 pressure cycle and record with 24 FPS at 20 mm from the border and record an image with the ball touching the wall.
	\item SHUT OFF THE LIGHT WHILE SAVING.
	\item Measure the temperature of the liquid.
	\item Shut off the rail pressure inlet.
	
	\item Open off the rail pressure inlet.
	\item Measure the temperature of the liquid.
	\item Perform a cycle of pressure with high FPS (to be determined), three times and record.
	\item SHUT OFF THE LIGHT WHILE SAVING.
	\item Measure the temperature of the liquid.
	\item Shut off the rail pressure inlet.
	
	\item Open off the rail pressure inlet.
	\item Measure the temperature of the liquid.
	\item Perform a ``pichnette'' experiment with spherical shape and record.
	\item SHUT OFF THE LIGHT WHILE SAVING.
	\item Measure the temperature of the liquid.
	\item Shut off the rail pressure inlet.
	
	\item Open off the rail pressure inlet.
	\item Measure the temperature of the liquid.
	\item Perform a ``pichnette'' experiment with buckled shape and record.
	\item SHUT OFF THE LIGHT WHILE SAVING.
	\item Measure the temperature of the liquid.
	\item Shut off the rail pressure inlet.
	
	\item Change the shell
\end{enumerate}

\paragraph{}
At the end of each liquid instance, take a sample of the liquid and label it.\\
When the iteration is finished, add the water volume to get the new viscosity and let it rest for an hour to make sure there are no bubbles.\\

\paragraph{}
Deadline: 19/09/16.
\newpage
\noindent
\textbf{Viscosity:} \\\\
\textbf{Ball thickness:}\\\\
\resizebox{\textwidth}{!}{
\begin{tabular}{|l|l|}
	\hline
  Step & Temperature\\
	\hline
	\multicolumn{2}{|c|}{STEP 1 } \\
	\hline
  Begin multi-cycle & \\
	\hline
	End multi-cycle & \\
	\hline
	\multicolumn{2}{|c|}{STEP 2 } \\
	\hline
  Begin uni-cycle 1 & \\
	\hline
	End uni-cycle 1 & \\
  \hline
	Begin uni-cycle 2 & \\
	\hline
	End uni-cycle 2 & \\
	\hline
	Begin uni-cycle 3 & \\
	\hline
	End uni-cycle 3 & \\
	\hline
	\multicolumn{2}{|c|}{STEP 3 } \\
	\hline
	Begin pichnette-cycle sphere & \\
	\hline
	End pichnette-cycle sphere & \\
  \hline
	Begin pichnette-cycle buckled & \\
	\hline
	End pichnette-cycle buckled & \\
	\hline
\end{tabular}
}



\end{document}

