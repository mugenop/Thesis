%!TEX root = ../dissertation.tex
\chapter{Introduction}
\label{introduction}

\newthought{Theoretical buckling pressure}
$$P_{critical} \propto E (\frac{d}{R})^2$$
\section{Motives of the up-scaling}
\paragraph{}
To investigate the possibility of swimming through shell buckling, we opted for a macroscopic scale study, using objects of few centimeters in diameter. This allows us to access a larger set of control parameters, such as the external radius $R$ and thickness $d$ of the shell, the material rigidity $E$ and dissipation coefficients. As a consequence, a more accurate visualization of the buckling dynamics is attained, not reachable at the microscopic scale.
\paragraph{}This strategy also allows a broader choice of experimental setups, giving access to quantities and time scales hardly reachable for a microscopic study, especially since the microscopic objects at our disposal, were manufactured in another laboratory with strong restrictions on the control parameters previously stated. The main inconvenience was the impossibility to produce a deformation cycle within the usable range of pressure (-1 bar, 1 bar).\\
For all these reasons, the up-scaling was required and a technique was developed to manufacture shells, in situ.
